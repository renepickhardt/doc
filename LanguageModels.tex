% Created 2014-04-25 Fri 18:06
\documentclass[11pt]{article}
\usepackage[utf8]{inputenc}
\usepackage[T1]{fontenc}
\usepackage{fixltx2e}
\usepackage{graphicx}
\usepackage{longtable}
\usepackage{float}
\usepackage{wrapfig}
\usepackage{rotating}
\usepackage[normalem]{ulem}
\usepackage{amsmath}
\usepackage{textcomp}
\usepackage{marvosym}
\usepackage{wasysym}
\usepackage{amssymb}
\usepackage{hyperref}
\tolerance=1000
\author{Heinrich Hartmann, Rene Pickhardt}
\date{\today}
\title{A mathematical theory of Language}
\begin{document}

\maketitle
\tableofcontents


\newcommand{\DEF}[1]{{\it #1}} % Definitions
\newcommand{\EQUIV}{\Leftrightarrow} % Equivalence

\newcommand{\IN}{{\mathbb N}}
\newcommand{\IR}{{\mathbb R}}
\newcommand{\ra}{\rightarrow}
\newcommand{\lra}{\longrightarrow}
\newcommand{\union}{\cup}
\newcommand{\isect}{\cap}
 
\section{Preliminaries}
\subsection{Spaces of Sequences}
\label{sec-1-1}

\newcommand{\SEQ}{\Sigma}
\newcommand{\len}{\text{\it length}}

Let A be a finite set. In Examples, this will be the set of characters
or the set of words. We define the set of Sequences over $A$ as 
\[ \SEQ A = \{ (a_1,\dots,a_l) \,|\, a_i \in A, l \geq 0 \}. \] 
We introduce the following notation:
\begin{itemize}
\item The empty seuqence is denoted by $\epsilon \in \SEQ A$.
\item The length function is denoted by
  \[ \len: \SEQ A \lra \IN_0,\quad (a_1,\dots,a_l) \mapsto l. \]
\item We have the following canonical decomposition by sequence length
\[ \SEQ A = A^0 \union A^1 \union A^2 \dots \]
and denote by $i_N: A^N \ra \SEQ A$ the inclusion of the length-$i$ sequences into $\SEQ A$.

\end{itemize}

\newcommand{\pr}[1]{{\pi_{#1}}}
\newcommand{\PR}[2]{{\pi_{#1}^{#2}}}
\newcommand{\undef}{\langle \mathrm{undef} \rangle}
%\newcommand{\undef}{\diamond}

Furthermore for each index $i \geq 0$ we have a projection
\[ \pr{i}: \SEQ A \lra A_+, \quad (a_1, \dots, a_l) \mapsto 
\begin{cases}
  a_i & l \geq i  \\
  \undef & l < i
\end{cases}
. \]
Here $A_+ := A \union \{ \undef \}$. This space allows
$\pr{i}$ to be defined on whole of $\SEQ A$.

Furthermore if $N \geq 0$ and $i \geq 1$ are given, we define the $i$-th $N$-gram
projection to be:
\[ \PR{i}{N}: \SEQ A \lra A^N_+, \quad (a_1, \dots, a_l) \mapsto
\begin{cases}
(a_{i+0},\dots,a_{i+N-1})  & l \geq i+N-1  \\
\undef                   & l < i+N-1
\end{cases}.\]

Note that we get back $\pr{i}$ as $\PR{i}{1}$. Moreover $\PR{i}{0}$ is the
canonical projection of $\SEQ A$ to the one point space $A^0 \subset
A^0_+$.

We have the following two filtrations of $\SEQ A$
\[ A^0 = \SEQ^{\leq 0} A \subset \SEQ^{\leq 1} A \subset \SEQ^{\leq
  2} A \subset \dots \subset \SEQ A \]
and 
\[ \SEQ A = \SEQ_{\geq 0} A \supset \SEQ_{\geq 1} A \supset \SEQ_{\geq
  2} A \supset \dots \supset \bigcap \SEQ_{\geq i} A = \emptyset  \]
where 
\[ \SEQ^{\leq i} A = A^{0} \union \dots A^{i} \]
and
\[ \SEQ_{\geq i} A = A^{i} \union A^{i + 1} \union \dots \]

\subsection{Spaces of probability measures}
\label{sec-1-2}

\newcommand{\MES}{\mathcal{M}}  % Space of measures
\newcommand{\PROB}{\mathcal{P}}  % Space of probability measures
\newcommand{\EXP}[2]{E_{#1}[#2]} % Expectation of a random variable

Let $X$ be an at most countable\footnote{The assumption of
  countability could be dropped, at the expense of a slightly more
  technical treatment of infinite sums.} set. We denote by
\[ 
   \MES(X) = \{ \mu: X \lra \IR_{\geq 0} \,|\, \sum_{x \in X} \mu(x) <
\infty \}
\]
the space of all finite measures on $X$. For $A \subset X$ we write
$\mu[A] = \sum_{x \in A} \mu(x)$. This definition agrees with the
usual definition, of measures in the case of discrete spaces with
maximal $\sigma$-algebra.

The set of probability measures is defined as
\[ \PROB(X) = \{ \mu  \,|\,  \mu[X] = 1 \} \subset \MES(X). \]
We get a normailzation map,
\[ \MES (X) \setminus \{0\} \lra \PROB (X), \quad \mu \mapsto
\frac{1}{\mu[X]} \mu \]
which is defined for non-zero measures $\mu$.

Let $f:X \lra Y$ be a map of sets, then we get a natural map
\[ f_*: \MES(X) \lra \MES(Y), \quad
   f_*(\mu)(y) = \mu[ f^{-1}(\{y\})] = \sum_{x:f(x)=y} \mu(x) \]
as well as $f_*: \PROB(X) \ra \PROB(Y)$.

If $f$ has finite fibers, then we also have the following map:
\[ f^*: \MES(Y) \lra \MES(X), \quad \mu \mapsto \mu \circ f, \]
however the total volume of $Y$ is not preserved, so that no map on
$\PROB$ is induced. In particular case, that $\iota: A \subset X$ is
the inclusion of a subspace we write $\mu|_A$ for $\iota^*(\mu)$.  If
for $P\in \PROB(X)$ the restriction $P|_{A}$ is not necessary a
probability measure. If $P[A] \neq 0$, then $P|_A$ can be normalized
to \[ P[\_|A]=\frac{1}{P[A]}P|_A,\] the conditional probability
measure on $A$.

If $\mu \in \MES_f(X)$ and $g:X \ra \IR$, we define the \DEF{expectation}
of $g$ as:
\[ \EXP{\mu}{f} :=  \sum_{x \in X} g(x) \mu(x). \]
This sum is well defined since $\mu$ is  finitely supported.

\subsection{Probability measures on sequences}

In this section we study the relationship between $\PROB(A)$ and
$\PROB(\SEQ A)$.

For $i \geq 1$ and $N \geq 0$ we have the following maps:
\[ \pr{i}_*: \PROB(\SEQ A) \lra \PROB(A_+), \quad \PR{i}{N}_*: \PROB(\SEQ A) \lra \PROB(A^N_+), \]
as well as
\[ \len_*: \PROB(\SEQ A) \lra \PROB(\IN_0). \]

\subsubsection{from Language Models to unigram measures}

Hence, for each probability measure $P$ on $\SEQ A$, we have
$\pr{i}_* P$ which is a measure on $A_+$. Note that
\[ \pr{i}_* P (\undef) = P[\{s \,|\, \len(s) < i \}] = P[\len < i]. \]
In the case, that $\pr{i}_* P (\undef) \neq 1$ we can normailze
$\pr{i}_* P$ to the $i$-th element distribution
\[ D_i P = \frac{1}{P[\len \geq i]} \, \pr{i}_* P = \pr{i}_*P[\_|\len
\geq i]  \;\in\; \PROB(A)\]
on $A$.

To define a total distribution of all elements, we want to take the
following infinite sum of $i$-th element distributions
\[ \sum_{i \geq 1} \pr{i}_* P \]
However, this sum is not necessarily finite for a general measure $P
\in \PROB(\SEQ A)$, therefore we make the additional assumption that $P$
is finitely supported and define
\[ M^1 : \MES_f(\SEQ A) \lra \MES(A), \quad \mu \mapsto 
         \sum_{i \geq 1} \pr{i}_* \mu. \]
Note that the measure is only defined on $A \subset A_+$, since
we have
\[ \sum_{i \geq 1} \pr{i}_*(\undef) = \sum_{i \geq 1} P[\len < i] =
\infty. \]
We calculate the total volume to be
\[ M^1 \mu(A) = \sum_{i \geq 1} \mu[\len \geq i] = \EXP{\mu}{\len}. \]
Hence, for $P \in \PROB_f(\SEQ A)$ with $P[\len = 0] \neq 1$, we can
normalize the measure $M^1P$, and define \DEF{unigram distribution} on
$A$ as
\[ D^1 P := \frac{1}{\EXP{P}{\len}} \sum_{i \geq 1} \pr{i}_* P \;\in\;
\PROB(A). \]

%%%%%%%%%%%%%%%%%%%%%%%%%%%%%%

\subsubsection{From unigram measures to Language Models}
Let $P \in \PROB(A)$ be a probability measure. 
$\forall l \in \IN$ we can pull back $P$ via $\pi_i$ to $P(A^l)$ and define
\[B_l^1P = \prod_{i=1}^l \pi_i^* P = \prod_{i=1}^l P \circ \pi_i \]
Explicitly for $s = (a_1,\dots,a_l) \in A^l$ this means $P(s)=\prod_{i=1}^l P(\pi_i(s))$.
Note that in these cases $\pi_i(s) \in A$ is well defined.

In the case of $l=1$ it is trivial to see that we receive a probability measure.
For $l=2$ we have:
\[\sum_{s\in A^2}B^1_2P = \sum_{s\in A^2} \prod_{i=1}^2 P(\pi_i(s)) = \sum_{a_1,a_2} P(a_1)P(a_2) = \sum_{a_1} P(a_1) \sum_{a_2} P(a_2) = \sum_{a_1} P(a_1) = 1\]
A similar argument will hold for all $l \in \IN$ showing that the pulled back probability measure $B_l^1P$ is indeed a probability measure on $A^l$. 

In order to construct a measure in $\PROB(\SEQ(A))$ starting from $P\in \PROB(A)$ we have a lot of choice (indicating that $\PROB(\SEQ(A))$ is indeed bigger than $\PROB(A)$).
Note that simply adding up $B_l^1P$ by setting 
\[B_{naiveTry}P = \sum_{l\geq 1}B^1_lP\] 
will not work as 
\[\sum_{s\in \SEQ A}B_{naiveTry}P(s) = \infty \neq 1\]
We can simply fix this by weighting the sum with an arbitrary chosen probability distribution $P_{weight}\in \PROB(\IN)$.
So we receive a Langugage Model from a Unigram Model by setting:
\[B^1P = \sum_{l\geq 1}  P_{weight}(l)B^1_lP = \sum_{l\geq 1} P_{weight}(l)\prod_{i=1}^l P \circ \pi_i \]

When applying statistics one could estimate the length distribution on sentences as a weighting distribution.
\textbf{We think one should investigate smoothing techniques for language models by changing this choice}
Another open end which I did not include yet is the idea of pushing forward $B^1_lP$ via $i_N:A^N \lra \SEQ(A)$. Well I think I did this implicetly by not being totally clean when using $\pi_i$ of stating if it was defined on $\SEQ A$ or $A^l$.
\textbf{is it possible to show that the above mentioned choice is up to a probability measure from $\PROB(\IN)$?}

%Before proofing that these maps are probability measures we will use the more formal notation from above to formulate the statement. 



%%%%%%%%%%%%%%%%%%%%%%%%%%%%%%

\subsubsection{From Language Models to $N$-gram measures}

More generally, for integers $i \geq 1$ and $N \geq 0$ we get a
measure $\PR{i}{N}_*(P)$ on $A^N_+$, for which
\[ \PR{i}{N}_* P (\undef) =  P[\len < i + N -1]. \]
If this number is not equal to one, we define the $i$-th $N$-gram
distribution to be
\[ D_i^N P = \frac{1}{P[\len \geq i+N-1]} \, 
             \PR{i}{N}_* P \;\in\; \PROB(A^N) \]
For the global $N$-gram distributions we take
\[ M^N: \MES_f(\SEQ A) \lra \MES(A^N), \quad \mu \mapsto \sum_{i
\geq 1} \PR{i}{N}_* \mu. \]
Again, this measure is only defined on $A^N \subset A^N_+$.
We calculate the total volume as
\begin{align} 
  M^N(\mu)(A^N) &= \sum_{i \geq 1} \mu[\len \geq i + (N-1)] \\
                &= \sum_{j = N} (j-(N-1)) \mu[\len = j]    \\
                &= \EXP{\mu}{\max\{ 0, \len - (N-1) \} }  
\end{align}
Note that this number depends only on the length distribution
$\len_*(\mu) \in \MES_f(\IN_0)$ and is non-zero if and only if
$\mu[\len \geq N] \neq 0$. 

Hence, for $P \in \PROB_f(\SEQ A)$, with $P[\len \geq N] > 0$ we can define the
{\it total N-gram distribution} as
\[ D^N P = \frac{1}{\EXP{}{\max\{ 0, \len - (N-1) \} }}
           \sum_{i \geq 1} \PR{i}{N}_* P \;\in\; \PROB(A^N). \]
Note, that the special case of $N=1$ reduces to our earlier definition.

\subsubsection{From $N$-gram measures to Language Models}
\textbf{this section is work in progress}
We have to see if a similar argument holds to construct a Language Model from and $N$-gram measure. 
In this case we would get a map $B^N : \PROB(A^N) \lra \PROB(\SEQ)$

\newcommand{\BACK}[1]{{\partial^{#1}}}

\subsubsection{converting $N$-gram measures to $M$-gram measures}
\textbf{this section is work in progress}
We achieved by the above observations the following. 
We can now convert any given Language Model to an $N$-gram Model via $D^N$. We can also convert any given $M$-gram Model to a Language Model via $B^M$.
This can be seen from this Diagram:
\[ \PROB(A^N) \overset{B^N}{\lra} \PROB(\SEQ A) \overset{D^M}{\lra} \PROB(A^M) \]
In particular we can look at the case $M=N-1$:
For $P \in \PROB(A^N)$ we get $D^{N-1}B^NP$ the induced back off model.
We remark that the back off model is not well defined since $B^N$ is only defined up to a distribution from $\PROB(\IN)$.
We still define the backoff operator as $\BACK{N}:=D^{N-1} \circ B^{N}$ in this way we get a sequence of $N$-gram models.
\[ \dots \overset{\BACK{N+1}}{\lra} \PROB(A^{N}) \overset{\BACK{N}}{\lra} \PROB(A^{N-1}) \overset{\BACK{N-1}}{\lra} \dots \overset{\BACK{2}}{\lra} \PROB(A)  \]

\textit{discuss what happens in the boundary case when applying the backoff operator to a unigram model. Also discuss if the index should have been shifted by one.}
\textit{I am not to happy with choosing partial as a symbol. There are several reasons. 1.) it was used for skips with a subindex 2.) we might want to define something like $\partial_M^N$ for converting between models. 3.) the resulting sequence is not really a complex since applying $\partial$ twice in general will not result to the null map (also the spaces are not abelian groups as far as I understand) 3.) $\partial$ as well as $B$ don't reflect the arbitrary choice that was being made}
\subsection{Markov measures}

A probability measure $P$ on $\SEQ A$ is called $K$-Markov if
for all $l \geq K$, $b_0,\dots,b_l \in A$ and $n > l$ the condition
\begin{align*}
  &  P[ \pr{n} = b_0 \,|\, \pr{n-1} = b_1, \dots, \pr{n-K} = b_K, \dots, \pr{n-l} = b_l ] \\  
 \;=\;&  P[ \pr{n} = b_0 \,|\, \pr{n-1} = b_1, \dots, \pr{n-K} = b_K ]
\end{align*}
holds whenever both sides are well-defined, i.e.
$P[\pr{n-1} = b_1,\dots, \pr{n-l} = b_l ]$ is non-zero.

\subsubsection{The case of $\undef$}
The above definition, does not specify a
condition in the case one ore more of the $b_i$ are $\undef$.
For $b_0=\undef$ is unproblematic. In the case that $b_0 \in A$ and $b_j =
\undef$ for one $j > 0$, the condition is empty since $\pr{i-j} =
\undef$ implies $\pr{i}=\undef$. For the remaining case 
of $b_0 = \undef$ and $b_j = \undef$ for one or more $j > 0$,
the naive-condition does not extend. To see this, we take $K=0$ and
$l=1$ with $b_1 = \undef$, so that the extended condition reads
\begin{align}
P[\pr{n} = \undef \,|\, \pr{n-1} = \undef ] = P[\pr{n} = \undef ].
\end{align}
This implies $P[\pr{n-1} = \undef] = 1$, which does not always hold.

\subsubsection{0-Markov measures}
In the special case of $K=0$ we find
\begin{align*}
  P[\pr{n} = b_0 ] &= P[ \pr{n} = b_0 \,|\, \pr{n-1} = b_1] \\
  \EQUIV \quad P[ \pr{n} = b_0,  \pr{n-1} = b_1 ] 
                   &= P[ \pr{n} = b_0 ] P[\pr{n-1} = b_1 ]
\end{align*}
which is the definition of $P$-independent between $\pr{n}$ and
$\pr{n-1}$ random variables, except that the case $\undef$ is
excluded. We can account for that by using conditional probabilities.
Assume that $P[\len \geq n] > 0$ then, $\pr{n}, \pr{n-1}$ are
$P[\_|\len \geq n]$ independent random variables on $\SEQ_{\geq n} A$.

Similarly, we see that the full collection $\{ \pr{i} \}_{i \leq n}$ 
is $P[\_|\len \geq n]$-independent on $\SEQ_{\geq n} A$. Indeed, 
\[     P[ \pr{0} = a_0, \dots, \pr{n} = a_n | \len \geq n]
  = \frac{1}{P[\len \geq n]} P[ \pr{0} = a_0, \dots, \pr{n} = a_n ] \]
and
\begin{align*}
 P[ \pr{0} = a_0, \dots, \pr{n} = a_n] = &P[ \pr{0} = a_0, \dots, \pr{n} = a_n ] \\
  = & P[ \pr{n} = a_n     | \pr{n-1}=a_{n-1}, \dots, \pr{0} = a_0 ]  \\
    & P[\pr{n-1} = a_{n-1}| \pr{n-2}=a_{n-1}, \dots, \pr{0} = a_0 ]  \\
    & \dots \\
    & P[\pr{0} = a_{0}]
\end{align*}

\subsection{How to see smoothing}

\subsection{How to understand skipped $n$-grams}

\end{document}
